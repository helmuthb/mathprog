\documentclass[11pt, oneside, a4paper, fleqn]{article}
\usepackage{geometry}
\usepackage{amsmath}
\usepackage[english]{babel}
\geometry{a4paper,margin=30mm}
\usepackage{graphicx}	
\usepackage{amssymb}
\usepackage[hidelinks]{hyperref}
\usepackage{fancyhdr}
\usepackage{lastpage}
\usepackage{tikz}
\usepackage{tkz-graph}
\usepackage{listings}
\usepackage{siunitx}
\usepackage{arydshln}
\usepackage{enumitem}
\rhead{Helmuth Breitenfellner (08725866)}
\chead{}
\lhead{Mathematical Programming}
\rfoot{Page \thepage{} of \pageref{LastPage}}
\lfoot{Programming Exercise}
\cfoot{}
\pagestyle{fancy}
\renewcommand{\footrulewidth}{0.4pt}
\author{Helmuth Breitenfellner}
\title{Mathematical Programming\\
       Programming Exercise\\
       Part 1: Compact Formulations}
\date{}
\sisetup{
  round-mode          = places, % Rounds numbers
  round-precision     = 2, % to 2 places
}
\allowdisplaybreaks
\setlength\parindent{0pt}
\begin{document}
\maketitle
\begin{abstract}
This report describes the formulations of the $k$-nodes
Minimal Spanning Tree ($k$-MST).
The formulations are based on \textit{Single-Commodity Flows},
\textit{Multi-Commodity Flows} and the \textit{Sequential Formulation}
based on Miller, Tucker and Zemlin \cite{mtz1960}.
These formulations have then been implemented using \texttt{CPLEX}
in C++.

The implementation can be found on GitHub:\\
\url{https://github.com/helmuthb/mathprog/tree/master/ProgrammingExercise}

\end{abstract}
\section*{$k$-nodes Minimal Spanning Tree}

The $k$-MST task is to find the \emph{Minimal Spanning Tree} which
contains exactly $k$ nodes and minimizes the \emph{weight sum}
of the edges.

Existing formulations from the regular MST can be applied, but
some adjustment is necessary.

For the regular MST one can choose any node as the root of
the spanning tree, since each node will be contained.
This is in the literature usually the node $1$.
However, for the \emph{$k$-MST} this is not the case.
Therefore one introduces an artificial \emph{root node},
which has edges to all other nodes with weight $0$.

This root node is then used in the formulations where normally the
node $1$ is used.
Additional constraints are necessary to ensure that only
\emph{one} connection leaves the root node.

One has to allow for $k+1$ nodes in the resulting $k$-MST, as the
artificial root node will be contained and is to be
be removed from the results.

Also it requires additional variables to specify the nodes
selected - since not all nodes will be in the $k$-MST..

Most formulations performed better when based on a \emph{directed}
formulation.
Therefore, arcs where introduced, which exist for all edges in
both directions - except for the root node, which only has arcs
going out from it but not into it.

In all formulations the following values are predefined:
\begin{align*}
  n & \enskip ... \enskip \text{Number of nodes (\emph{Vertices})} \\
  m & \enskip ... \enskip \text{Number of edges} \\
  a & \enskip ... \enskip \text{Number of arcs}
\end{align*}

All formulations correspond 1-to-1 to the implemented
program which was then executed.
Constraints were introduced where try-and-error
indicated better performance - even if they were
redundant or stronger formulations seem possible.

\section*{Single-Commodity Flows Formulation (\emph{SCF})}

The following variables are introduced:
\begin{align*}
  z_i \in\{0,1\} & \enskip ... \enskip \text{Node $i$ is selected}\\
  x_{ij} \in\{0,1\} & \enskip ... \enskip\text{Edge $i,j$ is selected}\\
  y_{ij} \in\{0,1\} & \enskip ... \enskip\text{Arc $i,j$ is taken
             in the tree, going from $i$ to $j$} \\
  f_{ij} \in[0-k] & \enskip ... \enskip\text{\emph{Flow} from node $i$ to node $j$}
\end{align*}

With this variables one can define the \emph{Single-Commodity Flows}
formulation for the $k$-MST as follows:
\begin{align}
  f_{ij} & \le k\cdot y_{ij} \quad \forall i,j \ne 0 \\
  y_{ij} + y_{ji} & = x_{ij}\\
  x_{ij} & \le z_i, \enspace x_{ij} \le z_j \\
  f_{0i} & = k\cdot y_{0i}\\
  \sum_{j}f_{ij} - \sum_{j}f_{ji} & = -1\cdot z_i\quad \forall i \ne 0 \\
  \sum_{i,j}x_{ij} & = k \\
  \sum_{j} z_j & = k + 1 \\
  \sum_{i} y_{0i} & = 1
\end{align}

\section*{Multi-Commodity Flows Formulation (\emph{MCF})}

The following variables are introduced:
\begin{align*}
  z_i \in\{0,1\} & \enskip ... \enskip \text{Node $i$ is selected}\\
  x_{ij} \in\{0,1\} & \enskip ... \enskip\text{Edge $i,j$ is selected}\\
  y_{ij} \in\{0,1\} & \enskip ... \enskip\text{Arc $i,j$ is taken
             in the tree, going from $i$ to $j$} \\
  f^l_{ij} \in[0-1] & \enskip ... \enskip\text{\emph{Flow}
             of type $l$ from node $i$ to node $j$,
             $l \in \{1, ..., n\}$}
\end{align*}

With this variables one can define the \emph{Multi-Commodity Flows}
formulation for the $k$-MST as follows:

\begin{align}
  f^l_{ij} & \le y_{ij} \quad \forall l \ne 0 \\
  y_{ij} + y_{ji} & = x_{ij}\\
  x_{ij} & \le z_i \\
  x_{ij} & \le z_j \\
  \sum_{i}y_{ij} &= z_j \\
  \sum_{j}f^l_{ij} - \sum_{j}f^l_{ji} & = 0
           \quad \forall i \ne l, i \ne 0, l \ne 0 \\
  \sum_{j}f^i_{0j} & = z_i
           \quad \forall i \ne 0 \\
  \sum_{j}f^i_{ij} - \sum_{j}f^i_{ji} & = -1\cdot z_i
           \quad \forall i \ne 0 \\
  \sum_{i,j}y_{ij} & = k \\
  \sum_{i} x_{0i} & = 1
\end{align}

\section*{Sequential Formulation (\emph{MTZ})}

The Sequential Formulation was introduced by Miller, Tucker and
Zemlin in \cite{mtz1960}.

It is an extremely compact formulation (having only a small
number of conditions) but the LP-relaxation has a large Polyhedron
(see e.g. \cite{padberg1991}).

The basic idea is to give each node an \emph{order}
number, which denotes in which order it will appear in the
final $k$-MST.

The following variables are introduced:
\begin{align*}
  z_i \in\{0,1\} & \enskip ... \enskip \text{Node $i$ is selected}\\
  x_{ij} \in\{0,1\} & \enskip ... \enskip\text{Edge $i,j$ is selected}\\
  y_{ij} \in\{0,1\} & \enskip ... \enskip\text{Arc $i,j$ is taken
             in the tree, going from $i$ to $j$} \\
  u_{i} \in[0-k] & \enskip ... \enskip\text{\emph{Order}
             of node $i$ in the selected $k$-MST}
\end{align*}

With this variables one can define the \emph{Sequence (MTZ)}
formulation for the $k$-MST as follows:
\begin{align}
  y_{ij}+y_{ji} & = x_{ij} \\
  u_j & \ge u_i + 1 - (1 - y_{ij})\cdot k \\
  \sum_{j} y_{ji} & \le z_i \\
  x_{ij} & \le z_i \\
  x_{ji} & \le z_i \\
  \sum_{i,j} x_{ij} & = k \\
  \sum_{j} z_j & = k + 1 \\
  \sum_{i} x_{0i} & = 1 \\
  u_0 = 0
\end{align}

\section*{Implementation Details}

The implementation used the framework provided in the lecture.

Some adaptations include:
\begin{itemize}
\item Use subclasses of the \texttt{kMST\_ILP} class
      for the algorithms
\item Allow output of results with a \texttt{-v} option
\item Don't add callbacks (they come lated for \emph{CEC} and \emph{DCC} model)
\end{itemize}

Other than that the framework was (hopefully) used as supposed.

\section*{Output Table}

The optimizations have been executed multiple times -
100 times for the smaller graphs down to 5 times for the last graph -
and the median of the CPU time has been taken.
\emph{The other results were consistent for each round.}

The runtimes refer to the output from the program and indicate
CPU time for the solver - the actual time was of course larger.

The system used for the evaluation has an Intel\textregistered{}
Xeon\textregistered{} CPU E31245 @ 3.30 GHz with 32 GB RAM.

\begin{tabular}{cr|r|rrr|SSS}
\multicolumn{3}{c}{} &
% file & $k$ & Objective & 
  \multicolumn{3}{c}{BnB-Nodes} &
  \multicolumn{3}{c}{CPU Time} \\
file & $k$ & Objective & \emph{SCF} & \emph{MCF} & \emph{MTZ} & \emph{SCF} & \emph{MCF} & \emph{MTZ} \\
\hline
\texttt{g01.dat} &
  2 &
  46 &
  0 & 0 & 0 &
  0.02 &
  0.03 &
  0.02 \\
\texttt{g01.dat} &
  5 &
  477 &
  0 & 0 & 0 &
  0.02 &
  0.04 &
  0.02 \\
\hdashline
\texttt{g02.dat} &
  4 &
  373 &
  0 & 0 & 0 &
  0.04 &
  0.02 &
  0.03 \\
\texttt{g02.dat} &
  10 &
  1390 &
  0 & 0 & 0 &
  0.10 &
  0.10 &
  0.05 \\
\hdashline
\texttt{g03.dat} &
  10 &
  725 &
  0 & 0 & 0 &
  0.07 &
  0.42 &
  0.10 \\
\texttt{g03.dat} &
  25 &
  3074 &
  0 & 0 & 807 &
  0.14 &
  0.54 &
  0.56 \\
\hdashline
\texttt{g04.dat} &
  14 &
  909 &
  0 & 30 & 3 &
  0.15 &
  4.28 &
  0.17 \\
\texttt{g04.dat} &
  35 &
  3292 &
  0 & 0 & 355 & 
  0.22 &
  1.62 &
  0.43 \\
\hdashline
\texttt{g05.dat} &
  20 &
  1235 &
  0 & 0 & 186 &
  0.27 &
  3.19 &
  0.52 \\
\texttt{g05.dat} &
  50 &
  4898 &
  0 & 0 & 2390 &
  0.81 &
  3.17 &
  3.63 \\
\hdashline
\texttt{g06.dat} &
  40 &
  2068 &
  251 & 27 & 4957 &
  5.54 &
  116.41 &
  10.16 \\
\texttt{g06.dat} &
  100 &
  6705 &
  537 & 33 & 5077 &
  22.33 &
  267.45 &
  10.81 \\
\hdashline
\texttt{g07.dat} &
  60 &
  1335 &
  487 & 0 & 911 &
  20.37 &
  165.32 &
  11.16 \\
\texttt{g07.dat} &
  150 &
  4534 &
  974 & 0 & 1791 &
  404.69 &
  500.42 &
  18.04 \\
\hdashline
\texttt{g08.dat} &
  80 &
  1620 &
  462 & 0 & 998 &
  160.90 &
  406.22 &
  12.14 \\
\texttt{g08.dat} &
  200 &
  5787 &
  988 & 0 & 11894 &
  428.01 &
  9808.82 &
  33.49 \\
\end{tabular}

\section*{Interpretation}

Performance of SCF is quite good.
It could be used for problems of smaller and medium sized problems
scale within acceptable time.

Performance of MCF for smaller graphs is also good.
However, once graphs are larger than 100 nodes its performance
degrades very much.
Running the evaluation for the last graph took hours to complete.

\emph{It's quite possible that some tweaking with the constraints
could yield better results.}

The overall "winner" so far (if this were a competition) is the
MTZ formulation. Its compact size makes it a good choice for
both small and large graphs.

\subsection*{Branch-and-Bound Nodes}

The number of Branch-and-Bound Nodes can be seen as a measurement
of the strength of the Polyhedron of the LP-relaxation.
The smaller the Polyhedron is, the higher the probability that
no branch-and-bound is needed and that the best (LP) solution
is already integer.

This interpretation is generally consistent with the results.
MTZ is known to have a large Polyhedron for the LP-relaxation,
and consistently the number of BnB-nodes is largest for it.

MCF is known to have a very tight Polyhedron for the LP-relaxation,
and consistently the number of BnB-nodes is smallest for it.

However, due to the excess number of variables and conditions
its performance is degraded.

One surprise is the high number of BnB-nodes for the medium
sized graphs with the MTZ algorithm.
I do not have a conclusive explanation for this, other than
that CPLEX is a sophisticated piece of software which
uses a combination of optimization algorithms combined
with heuristics, and that these optimizations might influence the
number of BnB-nodes beyond the characteristics of the MILP.

\bibliographystyle{plain}
\bibliography{citations}
\end{document}  
